%-----------------------------------------------------------------------------%
\chapter{\babSatu}
\label{bab:1}
%-----------------------------------------------------------------------------%
Pada bab ini, akan dijelaskan tentang latar belakang dan permasalahan yang diselesaikan pada penelitian ini.


%-----------------------------------------------------------------------------%
\section{Latar Belakang}
\label{sec:latarBelakang}
%-----------------------------------------------------------------------------%
Indonesia memiliki sejarah yang panjang dan kaya sebagai sebuah negara. Sejak zaman kerajaan-kerajaan di Nusantara, kolonialisme bangsa Eropa, perjuangan kemerdekaan, hingga masa pasca kemerdekaan, Indonesia mengalami berbagai peristiwa dan melibatkan para tokoh sejarah yang penting. Akan tetapi, sebagian data sejarah yang mencakup peristiwa dan tokoh, terutama pada periode pra-kemerdekaan dan pasca-kemerdekaan termasuk Orde Lama dan Orde Baru, seringkali tersebar di berbagai tempat penyimpanan, termasuk arsip nasional dan arsip internasional milik Belanda dan Australia dengan penyajian yang terbatas. Kondisi data sejarah yang terpisah-pisah dengan keterbatasan penyajian ini menimbulkan tantangan bagi individu, khususnya sejarawan, yang ingin mengakses, menganalisis, dan mengolah data sejarah Indonesia secara utuh.

Umumnya sumber data tersebut masih berbentuk data mentah dalam format teks atau gambar. Sejarawan perlu melewati proses kritik dan interpretasi untuk memberikan makna pada sumber - sumber tersebut. Hasil interpretasi ini umumnya diperuntukan untuk dikonsumsi oleh manusia sehingga diperlukan tahapan tambahan untuk mengubah format data tersebut ke dalam format yang dipahami oleh komputer

Untuk mengatasi permasalahan tersebut, solusi yang diusulkan adalah pembuatan ontologi dalam representasi knowledge graph (KG) yang berisi data sejarah Indonesia. Dalam ilmu komputer, ontologi adalah spesifikasi formal dan eksplisit dari konsep yang disepakati bersama dalam domain tertentu (Studer et al. 1998). Karakteristik tersebut memungkinkan manusia dan mesin untuk memahami data secara bersama-sama. Dengan representasi KG, ontologi dapat menyatukan dan menghubungkan data peristiwa dan tokoh sejarah dari berbagai sumber sejarah secara semantik sehingga data tersebut menjadi lebih harmonis dan memudahkan akses, pengolahan, dan analisis data yang lebih efisien seperti yang telah dilakukan pada ontologi tentang Perang Dunia 1 yang bernama WarSampo (Koho et al. 2020).

Berangkat dari informasi dan karakteristik yang dihasilkan oleh knowledge graph tersebut, data sejarah dapat disajikan secara visual dengan format yang bervariasi seperti dalam visualisasi linimasa, peta, ataupun graf secara langsung. Visualisasi ini dapat membantu para individu termasuk sejarawan dalam menggunakan dan memahami data sejarah Indonesia yang begitu berlimpah secara garis besar tanpa harus membaca keseluruhan dokumen.

Visualisasi berupa linimasa dipilih untuk dapat merepresentasikan hubungan keterkaitan antar peristiwa dalam konteks waktu (Karam. 1994). Visualisasi linimasa dapat membantu para individu untuk mencari tanggal dimulainya hingga tanggal berakhirnya suatu peristiwa, serta urutan kejadian antar peristiwa apabila terdapat sekumpulan peristiwa yang dibuka pada saat yang bersamaan.

\noindent\todo{Tentukan latar belakang dari penelitian Anda di sini (\f{background}).}


%-----------------------------------------------------------------------------%
\section{Permasalahan}
\label{sec:masalah}
%-----------------------------------------------------------------------------%
\noindent\todo{Sebutkan permasalahan penelitian Anda dari latar belakang tersebut.}

%-----------------------------------------------------------------------------%
\subsection{Definisi Permasalahan}
\label{sec:definisiMasalah}
%-----------------------------------------------------------------------------%
Berikut ini adalah rumusan permasalahan dari penelitian yang dilakukan:
\begin{itemize}
	\item Bagaimana cara membuat pertanyaan penelitian?
\end{itemize}
\noindent\todo{Tuliskan permasalahan yang ingin diselesaikan. Bisa juga berbentuk pertanyaan}

%-----------------------------------------------------------------------------%
\subsection{Batasan Permasalahan}
\label{sec:batasanMasalah}
%-----------------------------------------------------------------------------%
Berikut ini adalah asumsi yang digunakan sebagai batasan penelitian ini:
\begin{itemize}
	\item Salah satu batasannya adalah, ini hanya \f{template}.
\end{itemize}

\noindent\todo{Umumnya ada asumsi atau batasan yang digunakan untuk menjawab pertanyaan-pertanyaan penelitian diatas.}


%-----------------------------------------------------------------------------%
\section{Tujuan Penelitian}
\label{sec:tujuan}
%-----------------------------------------------------------------------------%
Berikut ini adalah tujuan penelitian yang dilakukan:
\begin{itemize}
	\item Untuk memberikan \f{template} yang dapat mempermudah skripsi orang lain.
\end{itemize}

\noindent\todo{Tuliskan tujuan penelitian Anda di bagian ini.}


%-----------------------------------------------------------------------------%
\section{Posisi Penelitian}
\label{sec:posisiPenelitian}
%-----------------------------------------------------------------------------%
\todo{
	Sebutkan posisi penelitian Anda. Ada baiknya jika Anda menggunakan gambar atau diagram.
	Template ini telah menyediakan contoh cara memasukkan gambar.
}

\begin{figure}
	\centering
	\includegraphics[width=0.4\textwidth]{assets/pics/makara.png}
	\caption{Penjelasan singkat terkait gambar.}
	\label{fig:research_position}
\end{figure}

\noindent\todo{Jelaskan \pic~\ref{fig:research_position} di sini.}


%-----------------------------------------------------------------------------%
\section{Langkah Penelitian}
\label{sec:langkahPenelitian}
%-----------------------------------------------------------------------------%
Berikut ini adalah langkah penelitian yang telah dilakukan:
\begin{enumerate}
	\item Tinjauan literatur \\
	      Pada tahap ini, dipelajari teori-teori yang terkait dengan penelitian ini untuk mendapatkan konsep dasar yang dibutuhkan dalam mencapai tujuan penelitian.
	\item Analisis implementasi dan kesimpulan \\
	      Pada tahap ini, digunakan studi kasus untuk analisis terkait kegunaan \f{template}.
	      Setelah melakukan analisis tersebut, ditarik kesimpulan keseluruhan dari penelitian ini.
\end{enumerate}


%-----------------------------------------------------------------------------%
\section{Sistematika Penulisan}
\label{sec:sistematikaPenulisan}
%-----------------------------------------------------------------------------%
Sistematika penulisan laporan adalah sebagai berikut:
\begin{itemize}
	\item Bab 1 \babSatu \\
	      Bab ini mencakup latar belakang, cakupan penelitian, dan pendefinisian masalah.
	\item Bab 2 \babDua \\
	      Bab ini mencakup pemaparan terminologi dan teori yang terkait dengan penelitian berdasarkan hasil tinjauan pustaka yang telah digunakan, sekaligus memperlihatkan kaitan teori dengan penelitian.
	\item Bab 3 \babTiga \\
	      Apa itu Bab 3?
	\item Bab 4 \babEmpat \\
	      Apa itu Bab 4?
	\item Bab 5 \babLima \\
	      Apa itu Bab 5?
	\item Bab 6 \kesimpulan \\
	      Bab ini mencakup kesimpulan akhir penelitian dan saran untuk pengembangan berikutnya.
\end{itemize}

\noindent\todo{Anda bisa mengubah atau menambahkan penjelasan singkat mengenai isi masing-masing bab. Setiap tugas akhir pasti ada yang berbeda pada bagian ini.}
